\documentclass[12pt]{article}
\usepackage{bussproofs, array, amstext, amssymb} 
\usepackage{mathtools, extarrows}
\usepackage{latexsym}
\usepackage{syntax}
\usepackage{tabularx}
\usepackage{flowchart}
\usetikzlibrary{arrows}
\usepackage{array,multirow}
\usepackage[section]{placeins}
\usepackage[bottom]{footmisc}
\usepackage[charter]{mathdesign}
\usepackage{tablefootnote}
\usepackage[nodayofweek,level]{datetime}
\usepackage[affil-it]{authblk}
\usepackage{hyperref}
\usepackage{algorithm,algpseudocode}
\usepackage{textcomp}
\usepackage{makecell}


\algrenewcommand{\algorithmiccomment}[1]{\hskip3em$\rhd$ #1}
\algdef{SE}[SUBALG]{Indent}{EndIndent}{}{\algorithmicend\ }%
\algdef{SE}[SWITCH]{Case}{EndCase}[1]{\textbf{case}\ #1\ \textbf{of}}{\algorithmicend\ \algorithmicswitch}%
\algtext*{Indent}
\algtext*{EndIndent}
\algtext*{EndCase}
\algtext*{EndFor}% Remove "end for" text
\algtext*{EndWhile}% Remove "end while" text
\algtext*{EndIf}% Remove "end if" text
\algtext*{EndFunction}% Remove "end if" text


\renewcommand{\labelitemii}{$\star$}
\newcommand{\midtilde}{\raisebox{0.5ex}{\texttildelow}}
\newcommand{\at}[0]{\ @\ }
\newcommand{\Adm}[0]{\mathrm{adm}}
\newcommand{\An}[0]{\mathrm{An}}
\newcommand{\Len}[0]{\mathrm{len}}
\newcommand{\Gnd}[0]{\mathit{gnd}}
\newcommand{\Prf}[0]{\mathit{prf}}
\newcommand{\Bch}[0]{\mathit{bch}}
\newcommand{\Sqts}[0]{\mathit{sqts}}
\newcommand{\Sqt}[0]{\mathit{sqt}}
\newcommand{\idf}[1]{\#_{#1}} 
% \newcommand{\fresh}[1]{\#_{\mathit{fresh}}({#1})} 
\newcommand{\fresh}[1]{\mathit{fresh}_{\#}({#1})} 
\newcommand{\mfi}[1]{\mathrm{mfi}({#1})} 
\newcommand{\Dec}[1]{\xRightarrow{#1}} 
\newcommand{\bprf}[0]{\begin{prooftree}}
\newcommand{\eprf}[0]{\end{prooftree}}
\newcommand{\axm}[1]{\AxiomC{$#1$}}
\newcommand{\unr}[1]{\UnaryInfC{$#1$}}
\newcommand{\bnr}[1]{\BinaryInfC{$#1$}}
\newcommand{\tnr}[1]{\TrinaryInfC{$#1$}}
\newcommand{\rlb}[1]{\RightLabel{#1}}
\newcommand{\limp}[0]{\to}
\newcommand{\liff}[0]{\leftrightarrow}
\newcommand{\sq}[1]{\text{`#1'}} 
\newcommand{\lfi}[0]{\mathrm{lfi}} 
\newcommand{\size}[0]{\mathrm{size}}
\newcommand{\adm}[0]{\mathrm{adm}}
\newcommand{\concat}{%
  \mathbin{{+}\mspace{-8mu}{+}}%
}

\makeatletter
\def\@maketitle{%
  \newpage
  \null
  \vskip 2em%
  \begin{center}%
  \let \footnote \thanks
    {\Large\bfseries \@title \par}%
    \vskip 1.5em%
    {\normalsize
      \lineskip .5em%
      \begin{tabular}[t]{c}%
        \@author
      \end{tabular}\par}%
    \vskip 1em%
    {\normalsize \@date}%
  \end{center}%
  \par
  \vskip 1.5em}
\makeatother

\title{The TESC Proof Format for First-Order ATPs}

\author{Seulkee Baek}
\date{}

\affil{Department of Philosophy, Carnegie Mellon University}
\affil{\textit{seulkeeb@andrew.cmu.edu}}

\begin{document}
\maketitle

\begin{abstract}
TESC is a machine-checkable, low-level proof format for first-order ATPs. 
Every step in a TESC proof is verifiable, including preprocessing steps like 
Skolemization and CNF normalization. TESC proofs are processed using the T3P 
tool, which generates TESC proofs from input TSTP solutions and checks them
with multiple backend verifiers for redundancy. TESC and T3P demonstrated
robust performance in a test using all eligible problems in the TPTP library,
successfully compiling 98.2\% (resp. 92.8\%) of TSTP solutions produced by 
Vampire (resp. E) and verifying all proofs in 3.27\% (resp. 2.51\%) of the 
time it takes to produce their corresponding TSTP solutions.
\end{abstract}

\section{Introduction}

The lack of a standard, machine-checkable proof format is a long-standing 
issue for first-order automated theorem provers (ATPs). 
Keeping a modern ATP bug-free while introducing new features is notoriously  
difficult \cite{}, and the increasing complexity of ATPs only makes it 
more risky to trust them without checkable proofs.
In addition, the lack of proofs also makes it difficult to \textit{use}
the output of ATPs; e.g., there are many proof assistants which 
solve first-order goals by calling external ATPs, but they are forced to 
either blindly trust the ATPs' answers \cite{} or perform redundant proof 
searches \cite{} because the ATP output cannot be directly translated into 
valid proofs for the proof assistants. 

These limitations make it clearly desirable to introduce an ATP proof format  
which lifts the burden of paintaking correctness checks from ATP development 
and improves the interface between ATPs and their client software. The five main
desiderata for for a prospective ATP proof format, as listed by Martin and Suda \cite{}, 
are as follows:
\begin{enumerate}
  \item Compatibility with all inference techniques used by major ATPs
  \item Coverage of ``unsound'' steps whose conclusions are not logically implied by their premises
  \item Efficiency of verification
  \item Ease of implementation in ATPs with small performance overhead
  \item Wide adoption by the ATP commmunity
\end{enumerate}










% The TSTP format \cite{zhang2004tstp} is supported by most ATP systems and standardizes their output syntax, but its lack of a semantic specification precludes any guarantees of mechanical verification. 


The \href{https://github.com/skbaek/tesc}{\color{blue}{Theory Extensible Sequent Calculus (TESC)}} 
is a new proof format for first-order ATPs that aims to fill this gap, with a particular emphasis on 2, 3, and 4. 
It supports 2 by including rules for introducing an extensible set of theory axioms, 
and ensures 3 by compiling high-level proofs down to a fully elaborated low-level format. 
4 is achieved by using TSTP files which most ATPs are already capable of generating, 
so it requires no implementation effort and acceptably small performance penalties.
It is hoped that these strengths will encourage adoption by ATP users and developers, leading to 5. 
1 is also important in the long run, but it is not the focus of this work as it has lower immediate priority.

% Every step in a TESC proof can be mechanically verified, all the way from 
% the input problem to the derivation of a contradiction; proof checking is fast and does not involve any search; for supported ATPs, it requires 
% no modification and imposes no performance overhead other than enabling TSTP output and using default proof search settings. 
% 
The following sections are organized as follows: 
section \ref{sec:rel-works} discusses how this project relates to existing work. 
Section \ref{sec:conventions} describes the conventions used throughout the paper.
Section \ref{sec:proof-calc} presents the underlying proof calculus of the format.
Section \ref{sec:format} gives the concerete TESC syntax. 
Section \ref{sec:t3p} discusses the T3P tool for generating and verifying TESC files. 
Section \ref{sec:example} shows a simple example of a TPTP problem and its TESC proof.
Section \ref{sec:results} presents the results of tests using problems from the TPTP library. 
Section \ref{sec:conclusion} summarizes what was accomplished and discusses future work.



\section{Related Works} \label{sec:rel-works} 

The aforementioned survey by Martin and Suda \cite{} provides a concise summary of the status quo regarding 
ATP proof formats. The testing and verification efforts of the Vampire development team \cite{} gives a good 
sense of the difficulty and importance of ensuring correctness for cutting-edge ATP systems.

The standardization of ATP input and output offered by the TPTP \cite{} and TSTP \cite{} formats is 
a critical prerequisite which makes the TESC format possible. 

The largest difference between TESC and existing approaches to ATP result verification is that TESC has 
direct support for preprocessing steps (e.g. Skolemization) whose conclusions are not logically implied 
by existing premises. TESC is most closely related to Ivy \cite{} and GDV \cite{}, and is especially 
similar to the former in that it uses ATPs to generate independently verifiable proof objects.
But both Ivy and GDV have limited support for preprocessing steps: GDV omits their verification entirely,
and Ivy prevents ATPs from using optimized preprocessing strategies because it relies on a formally verified 
preprocessor. 

Outside first-order logic, SAT solvers also used to suffer from a similar lack of standard checkable proof formats.
This problem was solved by the introduction of DRAT \cite{wetzler2014drat} and LRAT \cite{cruz2017efficient} formats,
whose convincing success was a direct inspiration for the development of the TESC format.

% \section{Conventions} \label{sec:conventions}
% 
% Unless explicitly indicated otherwise, we assume symbols to range over domains as follows: 
% 
% \begin{minipage}[t]{\dimexpr.5\textwidth-1em}
% \begin{align*}
%   b & \in {0,1}\\
%   i,j,k & \in \mathbb{N}\\ 
%   \nu & \in \text{formula names}\\
%   t & \in \text{terms}\\
%   \vec{t} & \in \text{lists of terms}\\
% \end{align*}
% \end{minipage}
% \begin{minipage}[t]{\dimexpr.5\textwidth-1em}
% \begin{align*}
%   f, g \in\ & \text{functions}\\
%   r \in\ & \text{relations}\\
%   \phi, \psi \in\ & \text{formulas}\\
%   \Gamma \in\ & \text{lists of formulas}\\
%   \Pi \in\ & \text{input problems}\\
% \end{align*}
% \end{minipage}
% 
% \textit{Formula names} are identifiers used to uniquely identify formulas in a given input problem. 
% For the purposes of this paper, it suffices to assume there is an countably infinite number of them and
% details of their syntax are omitted.
% 
% \textit{Lists} are comma-separated sequences. For any list $L$, $L[k]$ denotes the $k$th element (in zero-based numbering) of $L$, 
% and $\Len(L)$ is the number of elements in $L$. E.g., $(x,y,z)[1] = y$ and $\Len{(x,y,z)} = 3$. 
% 
% \textit{Input problems} are functions from formula names to formulas. They are abstract encodings of TPTP problem files. 
% For any input problem $\Pi$ that encodes a TPTP problem file $P$ and an arbitrary formula name $\nu$, we may assume that
% \begin{itemize}
%   \item If there is an annotated formula $f$ in $P$ with the name $\nu$, then $\Pi(\nu) = f$.
%   \item If no such annotated formula exists, then $\Pi(\nu) = \top$.
% \end{itemize}
% 
% For any inference rule $R$ that infers a sequent from another sequent, we say that $R$ is satisfiability-preserving only (SPO) if 
% \begin{itemize}
%   \item For any sequents $X$ and $Y$ such that $R$ infers $Y$ from $X$ and $X$ is satisfiable, $Y$ is also satisfiable
%   \item It is NOT generally the case that, for any sequents $X$ and $Y$ such that $R$ infers $Y$ from $X$, every model of $X$ is also a model of $Y$.
% \end{itemize}
% In other words, SPO inferences are satisfiability-preserving inferences that are not logically sound. A typical example of an SPO inference 
% is Skolemization.
% 
\section{Proof Calculus} \label{sec:proof-calc}

\begin{figure}
  \centering
  \begin{align*}
  f ::= &\ \sigma\ |\ \idf{k}\\
  t ::= &\ x_k\ |\ f(\vec{t})\\
  \vec{t} ::= &\ \cdot\ |\ \vec{t}, t\\
  \phi ::= &\ \top\ |\ \bot\ |\ f(\vec{t})\ |\ \lnot \phi\ |\ \phi \lor \chi\ |\ \phi \land \chi\ |\ \phi \to \chi\ |\ \phi \leftrightarrow \chi\ |\ \forall \phi\ |\ \exists \phi\\
  \Gamma ::= &\ \cdot\ |\ \Gamma, \phi
  \end{align*}
  \caption{Abstract syntax of TESC terms and formulas.}
  \label{fig:abst-syn}
\end{figure}
The syntax of terms and formulas for the TESC proof calculus is shown in Figure \ref{fig:abst-syn}. 
We let $f$ range over \textit{functors}, which are used as both function and relation symbols.
TESC makes no distinction bewteen the two types of symbols because it simplifies implementation 
and is otherwise harmless, as the type of a symbol can always be determined from its context.
The name `functor' is borrowed from the TPTP syntax terminology.

The symbol $\sigma$ ranges over \textit{plain functors}, which are the usual relation or function 
symbols of first-order logic. We assume that there is a suitable set of symbols $\Sigma$, and let 
$\sigma \in \Sigma$. Symbols of the form $\idf{k}$ are \textit{indexed functors}, where the number 
$k$ is called the \textit{functor index} of $\idf{k}$. Indexed functors are useful for reducing
the cost of introducing fresh functors: if you know the largest functor index $k$ that occurs in the 
environment, you may safely use $\idf{k+1}$ as a fresh functor without costly searches over a large 
number of terms and formulas. \footnote{Thanks to Marijn Heule for suggesting this idea.}
%This is especially useful since in TESC proofs, there is an easy method for determining the highest functor index of a sequent, as we will see shortly. 

We let $t$ to range over terms, $\vec{t}$ over lists of terms, $\phi$ over formulas, 
and $\Gamma$ over sequents. The number $k$ in variable $x_k$ is the variable's 
\textit{De Bruijn index} \cite{de1972lambda}; the use of De Bruijn indices allows 
quantifiers to be written without accompnying variables. E.g., the variable $x_1$ 
in formula $\forall \exists P(x_0,x_1)$ is bound by $\forall$. As usual, parentheses may be 
inserted for scope disambiguation, and the empty list operator `$\cdot$' may be omitted when 
it appears as part of a complex expression. E.g., the sequent $\cdot, \phi, \psi$ and term 
$f(\cdot)$ may be abbreviated to $\phi, \psi$ and $f$. For common binary predicates we take 
the liberty of using infix notations, e.g. writing $x = y$ instead of $=(x,y)$. 

\renewcommand\theadalign{bc}
\renewcommand\theadfont{\bfseries}
\renewcommand\theadgape{\Gape[4pt]}
\renewcommand\cellgape{\Gape[4pt]}

\begin{table}
    \begin{tabular}{ | c | c | c |}
      \hline

      \thead{Rule} & \thead{Conditions} & \thead{Example} \\

      \hline

      \makecell{\axm{\Gamma, A(b,\Gamma[i])} \rlb{$A$} \unr{\Gamma} \DisplayProof} &  &  
      \makecell{\axm{\phi \liff \psi, \phi \to \psi} \rlb{$A$} \unr{\phi \liff \psi} \DisplayProof} \\

      \hline

      \makecell{\axm{\Gamma, B(0, \Gamma [i])} \axm{\Gamma, B(1, \Gamma [i])} \rlb{$B$} \bnr{\Gamma} \DisplayProof} & & 
      \makecell{\axm{\phi \lor \psi, \phi} \axm{\phi \lor \psi, \psi} \rlb{$B$} \bnr{\phi \lor \psi} \DisplayProof} \\

      \hline

      \makecell{\axm{\Gamma, C(t, \Gamma [i])} \rlb{$C$} \unr{ \Gamma} \DisplayProof} & 
      \makecell{$\lfi(t) \le \size(\Gamma)$} & 
      \makecell{\axm{\lnot \exists f(x_0), \lnot f(g)} \rlb{$C$} \unr{\lnot \exists f(x_0)} \DisplayProof} \\

      \hline
      
      \makecell{\axm{\Gamma,\ D(\size(\Gamma), \Gamma [i])} \rlb{$D$} \unr{ \Gamma} \DisplayProof} & &
      \makecell{\axm{\exists f(x_0),\ f(\idf{1})} \rlb{$D$} \unr{\exists f(x_0)} \DisplayProof} \\

      \hline
      
      \makecell{\axm{ \Gamma, N(\Gamma [i])} \rlb{$N$} \unr{ \Gamma} \DisplayProof} & &
      \makecell{\axm{\lnot \lnot \phi, \phi} \rlb{$N$} \unr{\lnot \lnot \phi} \DisplayProof} \\

      \hline
      
      \makecell{\axm{ \Gamma, \lnot \phi} \axm{ \Gamma, \phi}  \rlb{$S$} \bnr{ \Gamma} \DisplayProof} &
      $\lfi(\phi) \le \size(\Gamma)$ &
      \makecell{\axm{\lnot f(\idf{0})} \axm{f(\idf{0})}  \rlb{$S$} \bnr{\cdot} \DisplayProof} \\

      \hline

      \makecell{\axm{ \Gamma, \phi} \rlb{$T$} \unr{ \Gamma} \DisplayProof} &
      \makecell{$\lfi(\phi) \le \size(\Gamma)$,\\ $\adm(\size(\Gamma), \phi)$} & 
      \makecell{\axm{=(f,f)} \rlb{$T$} \unr{\cdot} \DisplayProof} \\

      \hline

      \makecell{\axm{} \rlb{$X$} \unr{ \Gamma} \DisplayProof} & 
      \makecell{For some $i$ and $j$,\\ $\Gamma[i] = \lnot \Gamma[j]$} & 
      \makecell{\axm{} \rlb{$X$} \unr{\lnot \phi, \phi} \DisplayProof} \\
    
      \hline

    \end{tabular}

    \caption{TESC inference rules.}
    \label{tab:inf-rules}
\end{table}


The inference rules of the TESC calculus are shown in Table \ref{tab:inf-rules}.
The TESC calculus is a one-sided first-order sequent calculus, so having 
a valid TESC proof of a sequent $\Gamma$ shows that $\Gamma$ is collectively unsatisfiable.
The $A$,$B$,$C$,$D$, and $N$ rules are the \textit{analytic} rules. 
Analytic rules are similar to the usual one-sided sequent calculus rules, 
except that each analytic rule is overloaded to handle several connectives at once. 
For example, consider the formulas $\phi \land \psi$, $\lnot (\phi \lor \psi)$, 
$\lnot (\phi \limp \psi)$, and $\phi \liff \psi$. In usual sequent calculi, you would 
need a different rule for each of the connectives $\land$, $\lor$, $\limp$, and $\liff$ to 
break down these formulas. But all four formulas are ``essentially conjunctive'' 
in the sense that the latter three are equivalent to $\lnot \phi \land \lnot \psi$, 
$\phi \land \lnot \psi$, and $(\phi \limp \psi) \land (\psi \limp \phi)$.
So it is more convenient to handle all four of them with a single rule that analyzes 
a formula into its left and right conjuncts, which is the analytic $A$ rule.
Similarly, the $B$, $C$, $D$ rules are used to analyze essentially disjunctive, 
universal, existential formulas, and the $N$ rule performs double-negation elimination. 
For a complete list of formula analysis functions that show how each analytic 
rule breaks down formulas, see Appendix \ref{apdx:faf}. The analytic rules are 
a slightly modified adaptation of Smullyan's \textit{uniform notation} for analytic 
tableaux \cite{smullyan1995first}, which is where they get their names from. 

Of the three remaining rules, $S$ is the usual cut rule, and $X$ is the axiom or init rule. 
The $T$ rule may be used to add \textit{admissable} formulas. A formula $\phi$ is admissable 
in respect to a target theory $\mathbf{T}$ and sequent size $k$ if it satisfies the following condition:  
\begin{itemize}
  \item For any good sequent $\Gamma$ that is satisfiable modulo $\mathbf{T}$ and 
  $\size(\Gamma) = k$, the sequent $\Gamma, \phi$ is also satisfiable modulo $\mathbf{T}$. 
\end{itemize}
More intuitively, the $T$ rule allows you add formulas that preserve satisfiability. 
Notice that the definition of admissability, and hence the definition of well-formed 
TESC proofs, depends on the implicit target theory. This is the `theory-extensible' part of TESC. 
In other words, TESC is closer to a \textit{scheme} for proof formats than to a single, fixed proof format: 
users can obtain specific \textit{implementations} of TESC by defining the $\mathrm{adm}$ predicate, thereby 
fixing the set of admissable formulas that may be introduced via the $T$ rule. The implementation must also 
include a decision procedure that determines the truth of $\mathrm{adm}(k, \phi)$
for any input value of $k$ and $\phi$ in order to make the resulting format mechanically verifiable.  

When there is a need to distinguish the general scheme from the concrete instance, we refer to
the current implementation of TESC as TESC-0. In TESC-0, $\mathrm{adm}(k, \phi)$ holds if and only if 
$\phi$ is either a trivial logical constant ($\top$ or $\lnot bot$), one of the three equality axioms 
(reflexivity, symmetry, or transitivity), or has one of the following forms:
\begin{itemize}
    \item $\forall \forall (x_1 = x_0 \to \ldots \forall \forall (x_1 = x_0 \to f(x_{2k+1},\ldots,x_1) = f(x_{2k},\ldots,x_0))\ldots)$, 
      where $f$ is any functor and $2k$ is the total number of $\forall$s
    \item $\forall \forall (x_1 = x_0 \to \ldots \forall \forall (x_1 = x_0 \to (r(x_{2k+1},\ldots,x_1) \to r(x_{2k},\ldots,x_0)))\ldots)$,  
      where $r$ is any functor and $2k$ is the total number of $\forall$s
    \item $\forall \ldots \forall (\idf{k}(x_0, \ldots, x_m) \liff \phi)$, 
      where $\phi$ is any formula and $m$ is the number of outer universal quantifiers $\forall \ldots \forall$.
    \item $\forall \ldots \forall (\exists \phi \limp \phi[0 \mapsto \idf{k}(x_0, \ldots, x_m)])$, 
      where $\phi$ is any formula and $m$ is the number of outer universal quantifiers $\forall \ldots \forall$.
\end{itemize}
The first two takes care of congruence, the third handles new predicate definitions, and the last is
used for adding choice axiom. Note that TESC-0 has no special support for Skolemization other 
than the choice axioms. Therefore, any Skolemization step that is not an instance of the choice axiom 
given above (e.g., simultaneous elimination of multiple existential quantifiers) must be `spelled out' 
as the introduction and usage of multiple choice axioms.

Let us say that a sequent $\Gamma$ is \textit{good} if it satisfies $\lfi(\Gamma) < \size(\Gamma)$,
where $\lfi(x)$ denotes the largest functor index occurring in an expression $x$ (if $x$ incudes no 
functor indices, $\lfi(x) = -1$). All TESC inference rules are designed to preserve goodness; i.e.,
if the conclusion sequent of a rule is good, then so are all of its premise sequents. (we are concerned 
with preservation in the bottom-up direction, since this is the direction in which proofs are constructed 
and verified). Most rules ($A, B, N$, and $X$) trivially preserve goodness since they do not introduce 
any new terms or formulas. Others preserve goodness either by carefull choosing the newly introduced term
($D$) or enforcing it as a side condition ($C, S$, and $T$). The invariant that all sequents are good 
allows you to use the indexed functor $\idf{\Len(\Gamma)}$ whenever a fresh functor is needed for an 
inference (e.g., new predicate definitions).

\section{The Format} \label{sec:format}

The complete concrete syntax of the TESC proof format is given in Figure \ref{fig:tesc-syntax}. The syntax presentation is similar to 
that of the \href{http://www.tptp.org/TPTP/SyntaxBNF.html}{\color{blue}{TPTP syntax}}. 

The mapping between TESC syntax and the abstract syntax used in Section \ref{sec:proof-calc} is self-explanatory except for \texttt{<proof>}. 
In a \texttt{<proof>}, the first character always indicates the inference rule to be applied, followed immediately by arguments that determine 
how that inference rule should be applied, which in turn are followed by any remaining subproofs. E.g., in order to apply the $A$ rule, we need a 
boolean argument $b$ and a number $i$ in order to determine the new formula $A(b, \Gamma[i])$ to be added, so the leading character `A' is followed
by \texttt{<number>} and \texttt{<bool>}. For more details on how arguments for inference rule applications are read and used, see Section 
\ref{sec:format}.

One useful feature of the TESC syntax is that it is completely backtracking-free, because at any given point during parsing, each new 
character read unambiguously determines the syntactic class being parsed. This not only simplifies parser design, but also allows 
streaming verification where the verifier immediately parses and checks the prefix of proof already read without waiting for the 
remaining suffix.

\begin{figure}
    \centering
\begin{verbatim}

<hash> ::: [#]                <percent> ::: [%]
<dot> ::: [.]                 <dollar> ::: [$]
<comma> ::: [,]               <not_percent> ::: [^%]
<tilde> ::: [~]               <zero_numeric> ::: [0]
<vline> ::: [|]               <non_zero_numeric> ::: [1-9]
<ampersand> ::: [&]           <numeric> ::: [0-9]
<arrow> ::: [>]               <boolean> ::: [01] 
<equal> ::: [=]               <exclamation_mark> ::: [!]
<double_quote> ::: ["]        <question_mark> ::: [?]

<positive_decimal> ::= <non_zero_numeric><numeric>*
<decimal> ::= <zero_numeric> | <positive_decimal>
<number> ::= <decimal><percent>
<string> ::= <not_percent>*<percent>
<plain_functor> ::= <double_quote><string>
<indexed_functor> ::= <hash><number>
<functor> ::= <plain_functor> | <indexed_functor>
<variable> ::= <hash><number>
<term> ::= <variable> | <dollar><functor><term_list>
<term_list> ::= <dot> | <comma><term><term_list>
<connective> ::= <vline> | <ampersand> | <arrow> | <equal>
<quantifier> ::= <exclamation_mark> | <question_mark>
<formula> ::= <boolean>
            | <dollar><functor><term_list>
            | <tilde><formula>
            | <connective><formula><formula>
            | <quantifier><formula> 
<proof> ::= A<number><boolean><proof> 
          | B<number><proof><proof>
          | C<number><term><proof> 
          | D<number><proof>
          | N<number><proof>
          | S<formula><proof><proof> 
          | T<formula><proof>
          | X<number><number>
\end{verbatim}
    \caption{TESC syntax}
    \label{fig:tesc-syntax}
\end{figure}

\section{A Simple Example} \label{sec:example}

For a simple example of a TESC proof, consider the following problem (problem COL086-1 from the TPTP library): 

\begin{verbatim}
  cnf(fond_bird_exists,hypothesis,
      ( response(a,b) = b )).
  
  cnf(prove_happiness_2,negated_conjecture,
      (  response(a,B) != A )).
\end{verbatim}
A valid TESC proof for the problem, with line breaks and indentation added for legibility, is as follows:
\begin{verbatim}
C1%$"a%.
  C2%$"response%,$"a%.,$"a%..
    T!$"=%,#0%,#0%.
      C4%$"response%,$"a%.,$"a%..
        X3%5%
\end{verbatim}
Here's the same proof presented in a more human-readable form, where irrelevant formulas are abbreviated
and $\Gamma$ is the initial sequent that contains formulas in the input problem:
\bprf
\axm{}
\rlb{$X(3,5)$}
\unr{\Gamma,..., =(\text{response}(\text{a}, \text{a}), \text{response}(\text{a}, \text{a}))}
\rlb{$C(4,\text{response}(a,a))$}
\unr{\Gamma,..., \forall =(x_0, x_0)}
\rlb{$T(\forall =(x_0, x_0))$}
\unr{\Gamma, ..., \lnot =(\text{response}(\text{a}, \text{a}), \text{response}(\text{a}, \text{a}))}
\rlb{$C(2,\text{response}(a,a))$}
\unr{\Gamma, \forall \lnot =(\text{response}(\text{a}, \text{a}), x_0)}
\rlb{$C(1,a)$}
\unr{\Gamma}
\eprf



\section{The T3P Tool} \label{sec:t3p}

The TPTP-TSTP-TESC Processor (T3P) is the main tool for generating and verifying TESC proofs.
T3P accepts a TPTP problem and its TSTP solution as input, and compiles them to a TESC proof. 
Due to the wide variation of TSTP outputs between ATPs, there cannot be a universal 
TSTP-to-TESC compiler, and proof compilation must be implemented separately for each ATP. 
T3P currently supports compilation of solutions produced by Vampire and E in monomorphic 
first-order logic with equality. 

Since TESC requires explicit detail for all of its inferences, T3P must perform 
searches to fill in the information missing from TSTP solutions, such as positions 
of rewritten terms or pivot literals in resolution steps. These searches may 
occasionally blow up and cause compilation failures. For some types of inferences, 
search space explosion can be mitigated with custom-tailored solutions. E.g., 
T3P handles Vampire's AVATAR steps by exporting the problem to CaDiCaL \cite{queue2019cadical} 
and translating its proof output back to TESC. But many more complex inferences 
cannot be handled this way, so the best way to ensure fast and reliable compilation 
to TESC is generating fine-grained TSTP solutions with detailed information.

T3P also accepts a TPTP problem and its TESC proof as input, and verifies that the 
latter is a valid proof of the former's unsatisfiability. T3P includes three different
backend verifiers written in Rust, Agda, and Prolog that users can choose from.
T3P uses the Rust verifier by default, since it is optimized for performance and 
is most suitable for practical proof checking. The Agda verifier comes with a 
formal specification and proof of its soundness (whose details will be discussed in a 
separate upcoming paper) and is reasonably performant for most proofs,  
so it can serve as a fallback option when extra reliability is needed. 
The Prolog verifier is used mainly for debugging purposes.

Algorithm \ref{alg:tesc-check} shows a high-level description of a TESC verifier. 
Note that:
\begin{itemize}
  % \item The argument $\Prob$ is a dictionary which maps strings to formulas, and corresponds to the input problems (the component $\Pi$ 
  %   in a sequent $\Pi ; \Gamma$) we saw in Section \ref{sec:proof-calc}. The assumption is that $\Prob$ encodes the TPTP problem that 
  %   the input TESC proof is being verified against. 
  % \item The $\mathrm{LookUp}$ function retrieves formulas from $\Prob$ by the names they had in the input TPTP problem.
  \item The argument $\mathit{prob}$ is a list that holds the formulas of the input TPTP problem, 
    \textit{in the order} they originally appear in that problem. If a problem has \texttt{include} 
    clauses, the contents of included files are treated as inlined in the position of their \texttt{include} 
    clauses. The ordering is important because TESC inference rules identify formulas by their 
    positions in sequents.
  \item The use of $\mathit{prob}$ implies that TPTP formulas must be parsed into TESC formulas 
    before calling the main $\mathrm{Verify}$ function. This parsing is mostly straightforward, but there 
    are a couple caveats: 
    \begin{itemize}
      \item TPTP formulas in the CNF language are implicitly universally quantified, so the parser must 
        choose some ordering for the variables in a CNF formula before performing universal closure. 
        This ordering may be arbitrary but must be consistent, since a TESC proof generated assuming 
        one ordering and verified using another will fail to check. T3P sorts variables in a CNF formula by 
        their order of appearance in a left-to-write sweep: the earlier its appearance in the sweep, 
        the lower its De Bruijn index.
      \item The TPTP syntax includes logical connectives whose direct equivalents do not exist in TESC (e.g., \texttt{<\midtilde>}). During parsing, 
        they are replaced by suitable (combinations of) TESC connectives that preserve logical equivalence.
    \end{itemize}
  \item The argument $\Prf$ is the list of characters in the input TESC proof.
  \item The Verify function returns 1 if the proof is valid, and 0 otherwise.
  \item For each datatype $\mathrm{X}$, the fuction call ReadX($\Prf$) consumes as many characters as necessary from $\Prf$ 
    to parse and return a term of type $\mathrm{X}$. This also implicitly modifies $\Prf$ as the consumed characters are removed.
  \item Some functions used in Verify may fail. E.g., ReadBool($\Prf$) will fail if the first character of $\Prf$ is neither `0' nor `1'. We assume that the caller of Verify implements explicit failure handling such that, if Verify fails at any point, it is handled in the same way as when Verify returns 0.
  % \item The functions Adm, MaxFunctorIndex, and ApplyA correspond to adm, mfi, and (formula analysis function) A from Section \ref{sec:proof-calc}.
  \item Pop and Push adds an element to or removes an element from the beginning of a list.
\end{itemize} 


\begin{algorithm}[tb]
  \caption{TESC verification} \label{alg:tesc-check}
  \begin{algorithmic}[1]
  \Function{Verify}{$\mathit{prob},\Prf$}
    \State{$\Sqts \gets [\mathit{prob}]$} 
    \While{$\Sqts \neq []$}
      \State{$\Sqt \gets \mathrm{Pop}(\Sqts)$}
      \State{$c \gets \mathrm{Pop}(\Prf)$}
      \Case{$c$}
        \State{$\sq{A} \Rightarrow$}\Indent 
          \State{$i \gets \mathrm{ReadNumber}(\Prf)$}
          \State{$b \gets \mathrm{ReadBool}(\Prf)$}
          \State{$f \gets \mathrm{ApplyA}(b,sqt[i])$}
          \State{$\mathrm{Push}(\Sqt \concat [f],\Sqts)$}
        \EndIndent
        \State{$...$} \Comment{Other analytic rules are similar to A}
        \State{$\sq{S} \Rightarrow$}\Indent
          \State{$f \gets \mathrm{ReadFormula}(\Prf)$}
          \State{$\mathrm{Assert}(\mathrm{LargestFunctorIndex}(f) \le \mathrm{Length}(sqt))$}
          \State{$\mathrm{Push}(\Sqt \concat [f],\Sqts)$}
          \State{$\mathrm{Push}(\Sqt \concat [\lnot f],\Sqts)$}
        \EndIndent
        \State{$\sq{T} \Rightarrow$}\Indent
          \State{$f \gets \mathrm{ReadFormula}(\Prf)$}
          \State{$\mathrm{Assert}(\mathrm{LargestFunctorIndex}(f) \le \mathrm{Length}(sqt))$}
          \State{$\mathrm{Assert}(\mathrm{Admissible}(f))$}
          \State{$\mathrm{Push}(\Sqt \concat [f],\Sqts)$}
        \EndIndent
        \State{$\sq{X} \Rightarrow$}\Indent
          \State{$i \gets \mathrm{ReadNumber}(\Prf)$}
          \State{$j \gets \mathrm{ReadNumber}(\Prf)$}
          \State{$\mathrm{Assert}(sqt[i] =\lnot sqt[j])$}
        \EndIndent
        \State{$\_ \Rightarrow$ \Return $0$}
      \EndCase
    \EndWhile
    \State{$\_ \Rightarrow$ \Return $1$}
  \EndFunction
  \end{algorithmic}
  \end{algorithm}
  



\section{Test Results} \label{sec:results}

The performance of TESC and T3P was tested using all eligible problems from the TPTP problem library.
A given TPTP problem is eligible if it is (1) in the CNF or FOF languauge, (2) has 'theorem' or  
'unsatisfiable' status, (3) conforms to the official TPTP syntax, (4) assigns unique names to all of 
its formulas, and (5) is solved by Vampire or E under one minute. (3) is necessary because  
TESC and T3P assumes (per the official TPTP syntax) that the character `\%' does not appear in the
\verb|sq_char| syntactic class, and uses it as an endmarker when parsing \verb|sq_char|. 
(4) needs to be enforced in order to let T3P to unambiguously refer to premises during compilation.
There were a total of 7885 eligible problems for Vampire, and 4853 for E.

Add system information

Using the eligible problems, T3P successfully compiled 7792 and 4504 TSTP solutions 
generated by Vampire and E, respectively. All proofs produced were verified by T3P.
The difference between compilation success rates for Vampire (98.8\%) and E (92.8\%) 
can be largely attributed to solution granularity: TSTP solutions produced by E often 
chain together multiple inferences into a single step and omit the intermediate formulas, 
so T3P has to perform more search to fill in the missing information. 

\begin{table}[]
  \centering
  \begin{tabular}{l|cc}
                                    & Vampire   & E        \\ \hline
  Solution time (s)                 & 30220.478 & 8473.055 \\ 
  TSTP solutions size (MB)          & 735.981   & 53.841   \\
  Compilation time (s)              & 11796.303 & 5701.754 \\
  Compilation time / solution time  & 0.390     & 0.673    \\
  TESC proofs size (MB)             & 6203.458  & 479.737  \\
  Proofs size / solutions size      & 8.429     & 8.910    \\
  Verification time (s)             & 989.180   & 212.990  \\
  Verification time / solution time & 0.0327    & 0.0251  
  \end{tabular}
  \caption{Test results for TESC and T3P. Only the data for problems with successfully compiled TESC proofs are shown.}
  \label{tab:results}
\end{table}

Table \ref{tab:results} shows the test results for the problems whose solutions
were successfully compiled by T3P. The execution time results are 
encouraging, as it only takes a 39-67\% overhead to produce a checkable proof compared to the base
ATP solution time. Verification times are practically negligible compared to solution times, which
shows that TESC is efficient for re-use of ATP search results. The main downside of TESC is the 
>8 times increase in file size compared to TSTP, but this is unlikely to be a serious problem 
since ATP proofs are typically small (note that the set of \textit{all} TESC proofs that could be 
generated for the TPTP library still fits under 7 GBs) and ATPs usually exhaust
time/memory limits well before producing excessively large proofs.


\section{Conclusion}  \label{sec:conclusion}

In this paper, we have introduced the TESC proof format and the T3P tool for generating and verifying TESC proofs. 
TESC imposes minimal overhead on ATPs by offloading difficult elaboration work to T3P. 
TESC also allows mechanical verification of SPO steps by reducing them to applications of a small number of well-defined $T$ rules.
In a comprehensive test using Vampire and E on all eligible TPTP problems, we have  
demonstrated that this approach scales to practical problems and state-of-the-art ATPs.

The main limitation of TESC at the moment is the compilation bottleneck: implementation of 
a TSTP-to-TESC compiler is highly time-consuming, must be duplicated for each ATP, and the resulting compiler is 
likely to require frequent updates because it is sensitive to small changes in the ATP output.
The best solution to this problem is introducing a second intermediate format, most likely a subset of TSTP, 
that ATPs can generate directly and includes all the information
required for efficient compilation. Using a single source format makes it easier to 
optimize and maintain a proof compiler, and its similarity to TSTP will also make it 
easier for developers to support its output in ATPs.
Work on this intermediate format is already underway, and will be the next 
step in the development of TESC.


% its narrow coverage, both in terms of the number of ATPs it can support and the 
% logical fragments it can handle. Regarding the former, there seems to be two main ways forward: ideally, we may identify a subset 
% of TSTP that is expressive enough to be targeted by a diverse range of ATPs, but restrictive enough to allow compilation to TESC. 
% In this case, the universal TTC can be easily implemented and maintained, while ATP developers can ensure compatibility by targeting 
% the TSTP subset. It is an open research question, however, whether such a subset exists.
% 
% A slightly less ideal alternative is to recruit contributions from ATP developers for TSTP-to-TESC compilation. The current version
% of TTC was developed without any knowledge of Vampire and E's codebase, relying solely on studying their TSTP output. This is why 
% TTC so often resorts to brute-force search during compilation. If it is possible to achieve >90\% success rates by such indirect 
% methods, it is not unrealistic to expect near-perfect success rates with significantly improved speeds when the compilation strategy 
% for each ATP is designed by developers with deep understandings of their internal workings.
% 
% As for extensions to other fragments, the most realistic target would be support for many-sorted first-order logic, 
% which would also be practically useful considering its increasing popularity in ATPs. This is likely the next immediate goal 
% in TESC format development. Other potential targets, such as support for arithmetic, are more difficult and tentative.


\section*{Acknowledgements}  \label{sec:ack}

This work has been partially supported by AFOSR grant FA9550-18-1-0120.



\bibliographystyle{plain}
\bibliography{references}



\appendix

\section{Formula analysis functions} \label{apdx:faf}

\begin{centering}
    \begin{tabular}{|c|cc|}
    \hline
    A & 
    {
      $\begin{aligned}
        A(0, \lnot (\phi \lor \psi)) =&\ \lnot \phi\\ 
        A(0, \phi \land \psi) =&\ \phi\\ 
        A(0, \lnot (\phi \to \psi)) =&\ \phi\\ 
        A(0, \phi \liff \psi) =&\ \phi \limp \psi
      \end{aligned}$
    } &  
    {
      $\begin{aligned}
        A(1, \lnot (\phi \lor \psi)) =&\ \lnot \psi\\
        A(1, \phi \land \psi) =&\  \psi\\
        A(1, \lnot (\phi \to \psi)) =&\  \lnot \psi\\
        A(1, \phi \liff \psi) =&\  \psi \limp \phi
      \end{aligned}$
    } 
    \\ \hline
    B &
    {
      $\begin{aligned}
        B(0, \phi \lor \psi) =&\  \phi\\ 
        B(0, \lnot (\phi \land \psi)) =&\  \lnot \phi\\ 
        B(0, \phi \to \psi) =&\  \lnot \phi\\ 
        B(0, \lnot (\phi \liff \psi)) =&\  \lnot \phi \limp \psi
      \end{aligned}$
    } &  
    {
      $\begin{aligned}
        B(1, \phi \lor \psi) =&\  \psi\\
        B(1, \lnot (\phi \land \psi)) =&\  \lnot \psi\\
        B(1, \phi \to \psi) =&\  \psi\\
        B(1, \lnot (\phi \liff \psi)) =&\  \lnot \psi \limp \phi
      \end{aligned}$
    } 
    \\ \hline
    C & 
    $C(t, \forall \phi) = \phi [0 \mapsto t]$ &
    $C(t, \lnot \exists \phi) = \lnot \phi [0 \mapsto t]$
     \\ \hline
    D & 
    $D(k, \exists \phi) = \phi [0 \mapsto \idf{k}]$) &
    $D(k, \lnot \forall \phi) = \lnot \phi [0 \mapsto \idf{k}]$ 
    \\ \hline
    N & \multicolumn{2}{c|}{$N(\lnot \lnot \phi) = \phi$} 
    \\ \hline
  \end{tabular}
  
\end{centering}

\vspace{5mm}

    The Formula analysis functions used by analytic rules. For any argument not explicitly 
    defined above, the functions all return $\top$. E.g., $A(0, \phi \lor \psi) = \top$.
    $\phi[k \mapsto t]$ denotes the result of replacing all variables in $\phi$ bound 
    to the $k$th quantifier with term $t$. For the exact definition of substitutions 
    using De Bruijn indices, see Appendix \ref{apdx:subst}.


\section{Subtitution for TESC terms and proofs} \label{apdx:subst}

  \begin{itemize}
    \item $
      x_m [k \mapsto t] = 
      \begin{cases}
        x_m, & \text{if } m < k \\
        t, & \text{if } m = k \\
        x_{m-1}, & \text{if } m > k 
      \end{cases}
    $
    \item $(f(t_0, \ldots, t_m)) [k \mapsto t] = f(t_0[k \mapsto t], \ldots, t_m [k \mapsto t])$
    \item If $C \in \{\top, \bot\}$, then $C [k \mapsto t] = C$
    \item $(\lnot \phi) [k \mapsto t] = \lnot (\phi [k \mapsto t])$
    \item If $\bullet \in \{\lor, \land, \limp, \liff\}$, then $(\phi \bullet \psi) [k \mapsto t] = (\phi [k \mapsto t]) \bullet (\psi [k \mapsto t])$, 
    \item If $Q \in \{\forall, \exists\}$, then $(Q \phi) [k \mapsto t] = Q (\phi [k + 1 \mapsto incr(t)])$, 
      where $incr(t)$ is the result of incrementing all De Bruijn indices in $t$ by 1
    \item $(r(t_0, \ldots, t_m)) [k \mapsto t] = r(t_0[k \mapsto t], \ldots, t_m [k \mapsto t])$
  \end{itemize} 
  Note that the incrementation/decrementation of De Bruijn indices are necessary to ensure that variables are bound to the same quantifiers before and after substitution.

\end{document}

















